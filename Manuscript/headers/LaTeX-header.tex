\documentclass[11pt]{amsart} % this command specifies the font-size and that we want to use the AMS (American Mathematical Society) article class.
\usepackage{amssymb, amsmath, amsthm} %this command loads in special ways of formatting theorems and additional mathematical symbols.
\usepackage{color} %this package allows you to use colored text
\usepackage{graphicx} %this package lets you include images

%-----------------------------------------------
% Here are some optional packages you may want to use at some point. To make use of them uncomment them.

%\usepackage[shortalphabetic]{amsrefs}  %this package allows a certain kind of bibliography style. You won't need it for typical homework assignments.
 

\usepackage{fourier} %this command changes the font. You can find the commands for other fonts online. Be sure to use one that includes the math symbols. This font was chosen so that the blackboard bold "1" will show up correctly.
\usepackage[T1]{fontenc}



\usepackage[hidelinks=true]{hyperref} % this command turns references and citations into links.
%\usepackage{pinlabel} %this is useful package for adding labels to figures.
%-----------------------------------------------


%-----------------------------------------------
%Define theorem styles

\theoremstyle{theorem} % this sets the overall style for the following environments
\newtheorem{theorem}{Theorem} %this is the environment for theorems. To enter the environment you type \begin{theorem} as in the example below. The second argument is the name that appears when the document is compiled.       
\newtheorem*{theorem*}{Theorem} %This produces an unnumbered theorem
\newtheorem{lemma}[theorem]{Lemma} %This is the lemma enviornment. The middle argument specifies that Lemmas and Theorems should be given the same numbering.
\newtheorem{definition}[theorem]{Definition}
%-----------------------------------------------

\newcommand{\spacing}{\parskip 6.6pt \parindent 0pt} %This specifies that paragraphs should not be indented and that there should be a space between paragraphs
\spacing